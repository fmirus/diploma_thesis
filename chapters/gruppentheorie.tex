\chapter{Gruppentheorie}

Um die Galoisgruppe $G$ einer Differenzengleichung $L$ vom Grad
$n$ �ber dem Semi-Frobenius-K�roer $F$ mit zugeh�rigem
Frobenius-Modul $M$ zu bestimmen, ist es notwendig zun�chst einige
Infomrationen �ber die Gruppen, die als Galoisgruppe in Frage
kommen zusammenzutragen. Zun�chst wollen wir die m�glichen Gruppen
f�r Differenzengleichungen bestimmer Grade auflisten.
Anschliessend berechnen wir die Charaktere der symmetrischen
Produkte, um die Gruppen sp�ter auseinanderhalten zu k�nnen.

%$$\GL{2}{\galois{q}},\SL{2}{\galois{q}},\Or{2}{\galois{q}},\SO{2}{\galois{q}},\T{2}{\galois{q}},\Sp{2}{\galois{q}},\Di{2}{\galois{q}},\D{2}{\galois{q}},\D{\infty}{\galois{q}},\Gm{\galois{q}},\Ga{\galois{q}}$$
\section{M�gliche Gruppen}

Wir wissen nach Bemerkung \ref{1.14bem-galois-einbettung}, dass
die Galoisgruppe $G$ der Differenzengleichung $L$ isomorph zu
einer Untergruppe der $\GL{n}{F^{\phi}}$ ist. Wir wollen nun f�r
Gleichungen kleinen Grades $n=1,2$ die in Frage kommenden Gruppen
klassifizieren.

\subsection{Gleichungen vom Grad 1}
F�r Gleichungen vom Grad $1$ gibt es nur wenige M�glichkeiten f�r
die Galoisgruppe. Entweder es handelt sich um die volle
$\GL{1}{F^{\phi}}={F^{\phi}}^{\times}$ oder um endliche zyklische
Untergruppen von ${F^{\phi}}^{\times}$.

\subsection{Gleichungen vom Grad 2}

Bei Gleichungen vom Grad $2$ muss die Galoisgruppe eine
algebraische Untergruppe der $\GL{2}{F^{\phi}}$ sein.

\section{Die Charaktere der symmetrischen Produkte}
