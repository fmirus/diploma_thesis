 \newtheorem{bem}[equation]{Bemerkung}
 \newtheorem{vorbem}[equation]{Vorbemerkung}
 \newtheorem{satz}[equation]{Satz}
 \newtheorem{prop}[equation]{Proposition}
 \newtheorem{cor}[equation]{Korollar}
 \newtheorem{lemma}[equation]{Lemma}
 \newtheorem{zusatz}[equation]{Zusatz}
 \theoremstyle{definition}
 \newtheorem{defn}[equation]{Definition}
 \newtheorem{dca}[equation]{Decodieralgorithmus}
 \newtheorem{bda}[equation]{Basis-Decodieralgorithmus}
 \newtheorem{mda}[equation]{Modifizierter Decodieralgorithmus}
 \newtheorem{bspdef}[equation]{Beispiel und Definiton}
 \newtheorem{algo}[equation]{Algorithmus}
 \newtheorem{erinn}[equation]{Erinnerung}
 \newtheorem{bez}[equation]{Bezeichnung}
 \newtheorem{hyp}[equation]{Hypothese}
 \newtheorem{kon}[equation]{Konstruktion}

 \newtheorem*{konv}{Konvention}
 \newtheorem{anm}[equation]{Anmerkung}
 \newtheorem{notiz}[equation]{Notiz}
 \newtheorem{bsp}[equation]{Beispiel}
 \newtheorem{fr}[equation]{Frage}
 \newtheorem{aufg}[equation]{Aufgabe}
 \newenvironment{bew}{\begin{proof}[Beweis]}{\end{proof}}
 \newenvironment{bewanfang}{\textit{Beweis.}}{}
 \newenvironment{bewende}{\begin{center} \hfill \begin{math}\displaystyle }{\end{math} \hfill $\square$ \end{center}}

%Operatoren

 \DeclareMathOperator{\Char}{char}
 \DeclareMathOperator{\id}{id}
 \DeclareMathOperator{\Ord}{ord}
 \DeclareMathOperator{\Sym}{Sym}
 \DeclareMathOperator{\Auto}{Aut}
 \DeclareMathOperator{\Galoisgruppe}{Gal}
 \DeclareMathOperator{\diff}{d}
 \DeclareMathOperator{\tr}{Tr}
 \DeclareMathOperator{\SpSpur}{Spur}
 \DeclareMathOperator{\residuum}{res}
 \DeclareMathOperator{\Residuum}{Res}
 \DeclareMathOperator{\Der}{Der}
 \DeclareMathOperator{\Kern}{Kern}
  \DeclareMathOperator{\Kokern}{Kokern}
 \DeclareMathOperator{\Bild}{Bild}
  \DeclareMathOperator{\Kobild}{Kobild}
 \DeclareMathOperator{\ld}{ld}
 \DeclareMathOperator{\ggT}{ggT}
\DeclareMathOperator{\ggrT}{ggrT}
 \DeclareMathOperator{\ggt}{ggT}
 \DeclareMathOperator{\ur}{ur}
 \DeclareMathOperator{\kgV}{kgV}
 \DeclareMathOperator{\kgv}{kgV}
\DeclareMathOperator{\kglV}{kglV}
\DeclareMathOperator{\lmod}{mod}
 \DeclareMathOperator{\ERV}{ERV}
 \DeclareMathOperator{\Resul}{Res}
 \DeclareMathOperator{\Syl}{Syl}
\DeclareMathOperator{\Spek}{Spek}
\DeclareMathOperator{\Repr}{Repr} \DeclareMathOperator{\Vekt}{Vek}
\DeclareMathOperator{\Group}{Grp}
\DeclareMathOperator{\Algeb}{Alg}
 \DeclareMathOperator{\DFT}{DFT}
 \DeclareMathOperator{\re}{Re}
 \DeclareMathOperator{\rng}{Rang}
 \DeclareMathOperator{\Quot}{Quot}
 \DeclareMathOperator{\quot}{Quot}
 \DeclareMathOperator{\lead}{l}
 \DeclareMathOperator{\Lead}{L}
 \DeclareMathOperator{\LeadM}{LM}
 \DeclareMathOperator{\del}{\delta}
 \DeclareMathOperator{\inhalt}{c}
\DeclareMathOperator{\J}{J}
 \DeclareMathOperator{\sep}{sep}
\DeclareMathOperator{\ProjGenL}{PGL}
\DeclareMathOperator{\ProjSpecL}{PSL}
 \DeclareMathOperator{\GenL}{GL}
\DeclareMathOperator{\SpecL}{SL} \DeclareMathOperator{\Orth}{O}
\DeclareMathOperator{\SpecOrth}{SO}
\DeclareMathOperator{\Borel}{\korp{B}}
\DeclareMathOperator{\Diag}{Diag}
\DeclareMathOperator{\Torus}{\korp{T}}
\DeclareMathOperator{\Obj}{Ob} \DeclareMathOperator{\Morph}{Mor}
\DeclareMathOperator{\Homo}{Hom}
\DeclareMathOperator{\Mult}{\korp{G}_{m}}
\DeclareMathOperator{\Add}{\korp{G}_{a}}
\DeclareMathOperator{\Symp}{Sp} \DeclareMathOperator{\Died}{D}
\DeclareMathOperator{\End}{End} \DeclareMathOperator{\sol}{Sol}
\DeclareMathOperator{\Ten}{T} \DeclareMathOperator{\sign}{sign}
\DeclareMathOperator{\Ei}{E} \DeclareMathOperator{\height}{h}
\DeclareMathOperator{\DGal}{Gal} \DeclareMathOperator{\M}{M}
\DeclareMathOperator{\SGr}{S}
%\DeclareMathOperator{\id}{I}
 \newcommand{\Sol}[3]{\sol_{#2}^{#1}(#3)}
 \newcommand{\dime}[1]{\dim_{\galois{{#1}}}}
 \newcommand{\ord}[1]{\Ord_{#1}}
 \newcommand{\Spur}[2]{\SpSpur_{{#1}:{#2}}}
 \newcommand{\Tr}[2]{\tr_{{#1}:{#2}}}
 \newcommand{\Gal}[2]{\Galoisgruppe(\kerw{#1}{#2})}
 \newcommand{\Aut}[2]{\Auto(\kerw{#1}{#2})}
 \newcommand{\Res}[1] {\Residuum_{#1}}
 \newcommand{\res}[1] {\residuum_{#1}}
 \newcommand{\mf}[1] {\mathfrak {#1}}
\newcommand{\PGL}[2] {\ProjGenL_{#1}({#2})}
\newcommand{\PSL}[2] {\ProjSpecL_{#1}({#2})}
 \newcommand{\GL}[2] {\GenL_{#1}({#2})}
 \newcommand{\SL}[2] {\SpecL_{#1}({#2})}
\newcommand{\Or}[2] {\Orth_{#1}({#2})}
\newcommand{\SO}[2] {\SpecOrth_{#1}({#2})}
\newcommand{\B}[1] {\Borel_{2}({#1})}
\newcommand{\T}[1] {\Torus_{2}({#1})}
\newcommand{\Gm}[1] {\Mult({#1})}
\newcommand{\Ga}[1] {\Add({#1})}
\newcommand{\Sp}[2] {\Symp_{#1}({#2})}
\newcommand{\D}[2] {\Died_{#1}({#2})}
 \newcommand{\ratpkt}[1] {\mathbb{P}^{\; (1)}_{#1}}
 \newcommand{\inc}{\hookrightarrow}
 \newcommand{\fleck}{$\blacksquare$}
 \newcommand{\kerw}[2] {#1\negthickspace :\negthickspace #2} % KoerperERWeiterung
 \newcommand{\iso} {\cong}
 \newcommand{\var} {\mathcal{V}}
\newcommand{\Fas} {\underline{\Auto}^{\otimes}(\omega)}
 \newcommand{\Rep}[1] {\underline{\Repr}_{#1}}
\newcommand{\Vek}[1] {\underline{\Vekt}_{#1}}
\newcommand{\Alg}[1] {\underline{\Algeb}_{#1}}
\newcommand{\Grp}[1] {\underline{\Group}_{#1}}
\newcommand{\Hom} {\underline{\Homo}}
\newcommand{\Scheme}[1]{\underline{\Auto}^{\Phi}(#1)}
 \newcommand{\korp}[1] {\mathbb #1}
 \newcommand{\galois}[1] {\korp{F}_{#1}}
 \newcommand{\N} {\field N}
 \newcommand{\defind}[1]{\stich{#1}{\hervor #1}}
 \newcommand{\hooklongrightarrow}{\lhook\joinrel\longrightarrow}
 \newcommand{\twoheadlongrightarrow}{\relbar\joinrel\twoheadrightarrow}
 \newcommand{\abbildung}[6]{#1: \left\{ \begin{matrix} \ #2 & #3 & #4 \\ \ #5 & \longmapsto & #6 \end{matrix} \right.}
 \newcommand{\V}[1]{\textbf{\textit{#1}}} % Vektoren
 \newcommand{\hf}[1]{\textbf{\textup{#1}}} % Klassen von Divisoren, z.B. kanonische Klasse \hf{W}
\newcommand{\abb}[5]{\begin{array}{cccc}
     #1: & #2 & \longrightarrow & #3 \\
     & #4 & \longmapsto & #5\\
\end{array}}
\newcommand{\abbil}[3]{#1:  #2  \longrightarrow  #3 }
\newcommand{\abbger}[5]{#1:  #2  \longrightarrow  #3, \, #4 \longmapsto #5 }
\newcommand{\multi}[1]{\textbf{\textrm{\emph{#1}}}}
\newcommand{\Mo}[1]{\M(#1)}
\newcommand{\Moore}[2]{\Delta_{#1}(#2)}
\newcommand{\Ob}[1]{\Obj(#1)}
\newcommand{\Mor}[3]{\Morph_{#1}(#2,#3)}
\newcommand{\Div}[1]{(#1)_{\infty}}



%Zum Einheitlichen  Aendern aller Hervorhebungen
 \newcommand{\hervor}[1]{\textbf{\textup{#1}}}
%Randbemerkungen
 \newcommand{\rand}[1]{\marginpar{\tiny #1}}
%Indexeintraege
 \newcommand{\stich}[1]{\index{stichwort}{#1}}
 \newcommand{\zeich}[1]{\index{zeichen}{#1}}

